\section{First words}
\setauthor{Romeo Bhuiyan}
The development of artificial intelligence has opened numerous possibilities for various industries, including entertainment. 
One such field is the creation of Animotion, an AI-powered system that tracks and mimics the movements of the human face and 
body on a \GLS{vrm}. The purpose of this system is to allow users to create their own animated avatars that 
respond to real-world movements and interactions, making the virtual experience more engaging and interactive.

\section{Idea}
\setauthor{Romeo Bhuiyan}
The original intention behind the development of Animotion was to enhance the virtual reality experience by creating a more realistic representation of oneself. 
However, as the technology evolved, the possibilities for Animotion expanded as well. The AI-powered system not only tracks and mimics the movements of the 
human face and body on a VRM but also offers a social media platform that allows users to engage with each other in new and exciting ways.

The Animotion community is a vibrant and active group of users who enjoy performing dances, creating funny content, and uploading 
it to the Animotion media player. This allows others to view and interact with their content, further promoting the idea of community 
and collaboration. Users can also communicate with each other via Discord on the official Animotion community server, 
sharing tips and tricks for creating the best content. With Animotion, users have access to a wealth of provided animated avatars. These avatars 
respond to real-world movements and interactions, making the virtual experience more engaging and interactive. Users can 
also perform a wide range of actions with their avatars, such as dancing, posing, or performing other animations, 
adding even more depth to their virtual persona.

\section{Problems and Goals}
\setauthor{Romeo Bhuiyan}
\GLS{vr} is an increasingly popular technology that offers users a unique and immersive experience that allows them to interact with 
the digital world in new and exciting ways. However, the effectiveness of VR technology is often limited by the accuracy and efficiency of human 
movement tracking. This has been a major issue for many years, and while significant progress has been made in this area, there is still much 
work to be done. The problems with tracking human movements in VR stem from the limitations of the technology itself. In order to create a realistic virtual 
environment, it is essential that the movements of the user's face and body are accurately tracked and represented. This is necessary to 
create a sense of presence and immersion that is critical for an effective VR experience. However, current tracking technology is often prone 
to errors and inaccuracies, resulting in choppy and unrealistic representations of movement.

These inaccuracies can be caused by a variety of factors, such as poor lighting conditions, occlusions, and limitations in the 
hardware and software used to track movement. This can result in frustrating experiences for users who are trying to interact 
with the virtual environment, as their movements may not be accurately reflected in the virtual world.
Animotion seeks to address these problems by utilizing advanced AI algorithms to track and mimic the movements of the user's 
face and body. By leveraging the power of AI, Animotion can accurately track movement in real-time, resulting in a much more 
immersive and engaging VR experience. The AI algorithms used by Animotion are designed to analyze the movements of the user's face and body in real-time, capturing every 
subtle detail and nuance. This allows for a much more accurate representation of movement, resulting in a more lifelike virtual experience.
Additionally, Animotion offers a Discord-Server that allows users to connect and engage with each other in new and exciting ways. 
Users can perform dances, create funny content, and upload it to the Animotion media player for others to view and interact with. 
The AI-powered system tracks and mimics these movements, resulting in a more engaging and interactive virtual experience.
