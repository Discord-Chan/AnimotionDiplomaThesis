\section{Overview}
\setauthor{Lasinger Christoph}
The main focus of the project Animotion lies on the technical functionality 
of the translation of human face and body gestures captured by a camera to 
a VRM and the AI behind it. Therefore, the design of the website, on which
this functionality is displayed, mostly serves that focus. Besides the 
omepage, there are only four other webpages i.e., one showing the controlling 
of the model and the other three being minor ones, displaying less significant 
features.
\\
\\
\section{Homepage}
\setauthor{Lasinger Christoph}
As seen in figure \ref{fig:homepage} below, the homepage of the website starts with a big text
displaying the name of the project in a glitchy font with an even more glitchy
animation in order to immediately draw the user's attention. After that comes a
significantly smaller text in a less emphasizing colour, as it does not need to
be read by the user right away, simply giving instructions on what can be done
with the three images of 3D models below. As they stand in the centre of the page,
the three VRMs are on the first things one notices when visiting the website, which
is desired because clicking on one of them sends the user to the main page, where
the main functionality lies. Just for a pleasant effect, hovering over one of the
images displays a moving borderline, alternating between purple and pink. Below
each image is a button that navigates to another part of the website. Lastly, the
background for this page and the website in general is mostly black with one mirrored
girl on the side, mostly serving as nice accent and filling the rest of the page.
The main colour used for most design elements, be it the navigation buttons, main
fonts, back buttons, and so on is a nice-looking shade of purple (hex colour code
#8d4be0 to be exact). Less significant text, for example that of the bottom navigation
buttons, is simply white and any other text that is of moderate importance is mixture
between the two, created by taking the average of both colour values. 
\\
\\
\begin{figure}[htb]
    \centering
    \includegraphics[width=0.8\textwidth]{pics/Animotion_homepage.png}
    \caption{Homepage of the website}
    \label{fig:homepage}
\end{figure}
\\
\\
\section{Main page}
\setauthor{Lasinger Christoph}
After hovering over and eventually left clicking one image of the models on the homepage,
the user lands on the following page, shown in figure \ref{fig:mainpage} below. While the user takes in
the content of the page, the website quickly loads the model in the background. After it
finishes loading, it is displayed on the screen and the camera starts taking in information
and relaying it to the AI, which takes a short while. Once that process is done, an image
of the AI recognizes is laid over the camera window and the VRM starts moving according
to face and body gestures done by the user. In case the user wants to see a different, or
bigger part of the model, the controls are shown in the top right corner. The recording
feature and the controls to it can be found in the bottom left corner, where the user can
start or stop recording what is displayed on the website any time. Lastly, in the top left
corner is a button that removes any UI elements, that can in some situations be an unnecessary,
leaving the user to fully focus on the moving model and button that leads back to the homepage
to for example choose a different model. Because of the UI elements lingering in each corner,
the general background did simply not look good on this specific page and a different background,
that does not particularly draw any attention to it, was chosen.
\\
\\
\begin{figure}[htb]
    \centering
    \includegraphics[width=0.8\textwidth]{pics/Animotion_mainpage.png}
    \caption{Main page of the website}
    \label{fig:mainpage}
\end{figure}
\\
\\
\section{Community page}
\setauthor{Lasinger Christoph}
Here goes a description of figure \ref{fig:communitypage} below.
\\
\\
\begin{figure}[htb]
    \centering
    \includegraphics[width=0.8\textwidth]{pics/Animotion_community.png}
    \caption{Main page of the website}
    \label{fig:communitypage}
\end{figure}
\\
\\
\section{About us page}
\setauthor{Lasinger Christoph}
Here goes a description of figure \ref{fig:aboutpage} below.
\\
\\
\begin{figure}[htb]
    \centering
    \includegraphics[width=0.8\textwidth]{pics/Animotion_aboutus.png}
    \caption{About us page of the website}
    \label{fig:aboutpage}
\end{figure}
\\
\\
\section{Mediaplayer page}
\setauthor{Lasinger Christoph}
Here goes a description of figure \ref{fig:mediaplayerpage} below.
\\
\\
\begin{figure}[htb]
    \centering
    \includegraphics[width=0.8\textwidth]{pics/Animotion_mediaplayer.png}
    \caption{Mediaplayer page of the website}
    \label{fig:mediaplayerpage}
\end{figure}
\\
\\
