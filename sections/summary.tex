\section{Overview}
\setauthor{Lasinger Christoph}
In conclusion, the essential goal of this project, that is translating human face and body gestures onto 
a three-dimensional model using AI, was successfully accomplished in regard to performance and usability. 
As it currently mainly serves entertainment purposes, the observed enjoyable user experiences are a testament
to its success. Because the web application is being hosted on a server it is easily accessible under
\emph{animotion.live} which enables these fun experiences to be made anywhere by anyone at all times.

\section{Further work}
\setauthor{Lasinger Christoph}
In the future Animotion could be expanded by and in the following features:
\begin{itemize}
    \item Offering a wider choice of VRMs: Although the current model selection system of three images of VRMs
    being displayed on the home page of the website offers a clear and intuitive choice, it also imposes a limit
    of merely three models to choose from. By adding an extra menu or subpage these options could be expanded at
    the cost of simplicity.
    \item Featuring more content in the media player: As of now on the media player page there are only six videos 
    featured that we recorded ourselves. While the option of manually adding additional videos submitted by the 
    community does exist, it would be more convenient for users and to have an automated system for contributing content. 
    However, creating such a system would likely require much effort and should only be considered in case a large 
    community and with it a notably demand develops.
    \item Expanding and further developing the community: Establishing a new community from scratch requires hard work,
    lots of time and a significant amount of luck. As this not the main focus of Animotion but simply an idea for which
    the groundwork was laid, there has so far been little effort directed towards this prospect. This groundwork being
    the screen capture feature which can be used to record a user controlled VRM and the media player web page on which
    these records can then be displayed. Additionally, a discord server with text and voice channels was established for
    any potential Animotion community.
\end{itemize}

\section{Experiences}
\setauthor{Lasinger Christoph}
As the main focus of this project lies on the movement recognition AI, we naturally gained a lot of experience in developing
(working with?) not only AI in general, but specifically (this type of AI). By writing this thesis we acquired a deeper
theoretical insight as well. \\
Virtual, three-dimensional models were also new to us and by working on this project we enhanced our understanding in terms
of their limitations, variety and practicality regarding usage and obtainment. \\
Although we experienced some difficulties in the beginning regarding the use of web frameworks and incorporation of already
existing HTML and CSS, throughout the project we gained both theoretical and practical experience with different web frameworks
such as Angular, React, and Next.js.