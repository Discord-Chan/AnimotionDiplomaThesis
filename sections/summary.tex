\section{Overview}
\setauthor{Lasinger Christoph}
In conclusion, the essential goal of this project, that is translating human face and body gestures onto 
a three-dimensional model using AI, was successfully accomplished in regard to performance and usability. 
As it currently mainly serves entertainment purposes, the observed enjoyable user experiences are a testament
to its success. Because the web application is being hosted on a server and is easily accessible under 
\emph{animotion.live}, these fun experiences can made anywhere by anyone at all times.

\section{Homepage}
\setauthor{Lasinger Christoph}
In the future “Animotion” could be expanded by and in the following features:
\begin{itemize}
    \item Offering a wider choice of VRMs: Although the current model selection system of three images of VRMs being displayed on the home page of the website offers a clear and intuitive choice, it also imposes a limit of merely three models to choose from. This could be solved by
    \item Expanding and further developing the community
    \item Featuring more content in the media player
\end{itemize}

\section{Experiences}
\setauthor{Lasinger Christoph}
Although there were some ups and downs in the beginning regarding web frameworks and the incorporation of already
existing HTML and CSS, throughout the project we gained both theoretical and practical experience with web frameworks
such as Angular, React, and Next.js.
% Romeo gained experience and further interest in AI 