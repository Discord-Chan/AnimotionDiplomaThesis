




\section{Calculation of the virtual reality model} 
\setauthor{Romeo Bhuiyan}
Controlling a VRM on a website involves a complex process that involves 
capturing the user's inputs, updating the VRMs position and orientation in real-time, 
and rendering the updated model for display on the user's device.
The first step in this process is for the user to interact with the VR model 
using a device such as a VR headset or a mouse and keyboard. The inputs from the 
user are then captured by the website and sent to the server. The server then 
uses these inputs to update the VR model's position and orientation in real-time. 
\\
Once the VR model has been updated, it is then rendered, and the new frame is 
sent back to the user's device for display. This process repeats as the user 
continues to interact with the VR model. 
To achieve this, various technologies are needed such as GLTFLoader \cite{GLTFLoader} from the THREE 
libraries for rendering the VR model in the browser, WebSockets for real-time 
communication between the client and server, and a physics engine to simulate 
the movement and interactions of the VR model.


\subsection{}
{Romeo Bhuiyan}

