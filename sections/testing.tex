\section{Importance of testing}
\setauthor{Romeo Bhuiyan}
Usability testing is a crucial evaluation method that is used to assess the user 
experience of a web application. It involves having real users perform tasks on a 
website or application and observing their interactions, behaviors, and feedback. 
The goal of usability testing is to determine if the web application is user-friendly, 
efficient, and intuitive for the target audience.

Usability testing is important for web applications for several reasons. First and 
foremost, it allows developers and designers to identify any usability issues before 
the application is released to the public. This can save significant time and money, 
as fixing these issues later in the development cycle can be more time-consuming and costly.

In addition, usability testing helps ensure that the application meets the needs 
and expectations of the target audience. This is essential in ensuring that the 
application is successful and meets the needs of its users. Usability testing 
can also help identify areas for improvement, such as navigation, page layout, 
and content organization.

Another benefit of usability testing is that it provides valuable insight 
into how users interact with the application. This can help developers 
and designers understand how users approach tasks and what their pain 
points are when using the application. This information can then be 
used to make improvements to the user experience and ensure that 
the application is as user-friendly as possible.
\\
\\
\section{Functional testing}
\setauthor{Romeo Bhuiyan}
\textbf{Functional testing} is a type of software testing that verifies the functionality of a system, application, or a component. 
The main objective of functional testing is to ensure that the system meets the specified requirements and works as expected. 
It is a comprehensive evaluation of the software's functionality, covering all possible scenarios and use cases. 
This type of testing focuses on evaluating the features, capabilities, and end-to-end behavior of the system.

That involves testing each individual function or feature of the system in isolation, and then 
integrating and testing the system as a whole. The testing process starts with the requirement analysis, 
where the system requirements are reviewed, understood and prioritized. Based on the requirements, 
the test cases are then designed and executed. During the testing process, the tester performs 
various tests such as unit testing, integration testing, system testing, and acceptance testing.

Functional testing is an essential step in the software development life cycle, as it helps to 
identify any defects or bugs in the system early in the development process. This helps to 
reduce the cost of fixing the defects and ensures that the software is delivered to the 
end-users with high quality and reliability.

It helps to validate the software's compliance with the business requirements 
and user expectations, thereby ensuring the software's overall quality and usability. 
It is important to perform functional testing on a regular basis to ensure that 
any new changes or updates made to the system do not affect its functionality and performance.


\section{Non-functional testing}
\setauthor{Romeo Bhuiyan}
Wird noch \emph{smile}