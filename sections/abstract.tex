\begin{spacing}{1}
    \chapter*{Abstract}
\end{spacing}
Animotion is a combination of the two words animation and motion, cramming the general idea behind it into as little information as possible. This general idea being a camera, for example that of a laptop, a phone or an external one used in a computer setup, recording face and body gestures of the user and translating them onto a virtual reality model (VRM), thereby controlling it. This is done by using an artificial intelligence that calculates tracked gestures and puts them into a canvas that shows the default pose of the VRM and additional user input recorded by the camera. The aforementioned model can be selected by the user out of an assortment of three possible choices on the main page of the website. The moving VRM can be recorded, either the browser window or the entire screen, and for example posted on social media for entertainment purposes. Another possible application would be using Animotion as a way of recording yourself for your own livestream (“V-Tubing”), where, for a variety of reasons, one chooses to not represent oneself with one’s own body but a virtual (“fictional”) one instead.
\\
\\
The frontend, i.e., the website was mainly implemented using a combination of the JavaScript framework Next.js and Sass, a stylesheet language similar to CSS used for the general design. For the web application Mediapipe, an open-source, cross-platform framework used to build machine learning solutions for streaming media, and Holistic, a gesture analysis and control library of Mediapipe that is cropped to augmented reality effects, were chosen in order to depict the VRM together with three.js, a JavaScript library used to create and display 3D computer graphics in a web browser.
\\
\\
\\
\includegraphics[width=1\textwidth]{pics/animotionlogo.png}

\newpage
\begin{spacing}{1}
    \chapter*{Zusammenfassung}
\end{spacing}
Zusammenfassung unserer genialen Arbeit. Auf Deutsch.
Das ist das einzige Mal, dass eine Grafik in den Textfluss eingebunden wird.
Die gewählte Grafik soll irgendwie eure Arbeit repräsentieren.
Das ist ungewöhnlich für eine wissenschaftliche Arbeit aber eine Anforderung der Obrigkeit.
\emph{Bitte auf keinen Fall mit der Zusammenfassung verwechseln, die den Abschluss der Arbeit bildet!}
\\
\\
\\
\includegraphics[width=1\textwidth]{pics/animotionlogo.png}
